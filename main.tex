%master-thesis-summery
%#BIBTEX pbibtex main
\documentclass{jsarticle}

\title{Semilinear elliptic equations with critical Sobolev exponent and non-homogeneous term}
\author{Kazune Takahashi \thanks{Graduate School of Mathematical
Sciences, The University of Tokyo. Email: \texttt{kazune@ms.u-tokyo.ac.jp}}}
\date{28 January 2015}

\makeatletter
\renewcommand{\theequation}{%
\thesection.\arabic{equation}}
\@addtoreset{equation}{section}
\makeatother

\usepackage{geometry}
\geometry{a4paper,left=15mm,right=15mm,top=20mm,bottom=15mm}
\addtolength{\textheight}{-10mm}

\usepackage{newtxtext,newtxmath}

\usepackage{relsize}
\usepackage{url}
\usepackage{enumerate}
\addtolength{\labelsep}{1zw}
\newcommand{\sage}{
\addtolength{\itemindent}{1zw}
\addtolength{\parindent}{1zw}}

\usepackage{mathtools}
\usepackage{cases}

\usepackage{fancyhdr}
\pagestyle{fancy}
\lhead{}
\chead{}
\rhead{}
\lfoot{}
\cfoot{{\textit{\thepage}}}
\rfoot{}
\fancypagestyle{plainhead}{
\lhead{}
\chead{}
\rhead{}
}
\fancypagestyle{plainfoot}{
\lfoot{}
\cfoot{{\textit{\thepage}}}
\rfoot{}
}
\renewcommand{\headrulewidth}{0pt}
\renewcommand{\footrulewidth}{0pt}
\renewcommand{\sectionmark}[1]{\markright{\S~\thesection.~~#1}{}}

\usepackage{amsmath,amssymb}
\usepackage[amsmath,framed,thmmarks]{ntheorem}
\allowdisplaybreaks[1]

\theoremstyle{plain}
\theoremseparator{.}
\theorembodyfont{\upshape}
\theoremprework{}
\theorempostwork{}
\theoremnumbering{Alph}
\newtheorem{thm}{定理}

\usepackage{comment}

\newcommand{\pdif}[2]{\frac{\partial #1}{\partial #2}}

\renewcommand{\hat}[1]{\widehat{#1}}
\renewcommand{\tilde}[1]{\widetilde{#1}}
\renewcommand{\bar}[1]{\overline{#1}}
\renewcommand{\vec}[1]{\overrightarrow{#1}}

\newcommand{\N}{\mathbb{N}}
\newcommand{\R}{\mathbb{R}}
\newcommand{\Z}{\mathbb{Z}}
\newcommand{\C}{\mathbb{C}}
\newcommand{\Q}{\mathbb{Q}}

\renewcommand{\ae}{\text{a.\,e.}~}
\newcommand{\tin}{\text{in}~}
\newcommand{\ton}{\text{on}~}
\newcommand{\supp}{\operatorname{supp}}

\newcommand{\dnorm}{\left\|\, \cdot \, \right\|}

\renewcommand{\bibname}{参考文献}

\setcounter{tocdepth}{4}
\begin{document}

{\Large

\begin{center}
 \LARGE
 {\bf 論文内容の要旨}
\end{center}

\begin{description}
 \item[修士論文題目] Semilinear elliptic equations with critical
            Sobolev exponent and non-homogeneous term (ソボレフ臨界指数をもつ
            非斉次半線形楕円型偏微分方程式)
 \item[氏名] 高橋 和音 (Kazune Takahashi)
\end{description}
}

本論文では,以下の方程式を考察する.
\begin{align}
 \left\{
 \begin{aligned}
  -\Delta u + a u &= b u^p + \lambda f  & &\tin \Omega,  \\
  u &> 0 & &\tin \Omega, \\
  u &= 0 & &\ton \partial\Omega.
 \end{aligned}
 \right. \tag*{$(\spadesuit)_\lambda$} \label{eq:prob_main}
\end{align}
ここで,$N \geq 3$,$\Omega \subset \R^N$は有界領域,
境界$\partial \Omega$は$C^\infty$級とする.
$p = (N+2)/(N-2)$をソボレフ臨界指数とする.
$f \in H^{-1}(\Omega)$は,$f \geq 0$,
$f \not \equiv 0$をみたすとする.
$a, b \in L^\infty(\Omega)$とする.
$\kappa_1$を$-\Delta$の$\Omega$におけるディリクレ境界条件下の
第$1$固有値とする.$a \geq \kappa ~\tin \Omega$をみたす
$\kappa > - \kappa_1$が存在すると仮定する.
また,$b \geq 0 ~\tin \Omega$,$b \not \equiv 0$と仮定する.
$\lambda > 0$はパラメータである.

ソボレフ臨界指数を持つ半線形楕円型偏微分方程式の
正値解の存在・非存在は,
次元$N$や領域の形状に依存していることが知られている.
ブレジス -- ニレンベルグ\cite{MR709644}~が有名である.
その中でも,パラメータ$\lambda$が非斉次項につく方程式の研究が
なされている.
タランテッロ\cite{MR1168304}~では,
自明解を持たないある方程式に対し,
少なくとも$2$つの非自明正値解の存在が示された.

\ref{eq:prob_main}の minimal solution $\underline{u}_\lambda$とは
\ref{eq:prob_main}の任意の解$u$に対し,$u \geq \underline{u}_\lambda
~\tin \Omega$をみたすものである.minimal solution 以外の解を
便宜上 second solution という.
内藤 -- 佐藤~\cite{MR2886160}では,
\ref{eq:prob_main}において$a = \kappa$,$b = 1$としたものについて,
以下の事実が証明された.
\begin{enumerate}[1.] \sage
 \item 全ての$N \geq 3$,$\kappa > - \kappa_1$に対し,
       $0 < \bar{\lambda} < \infty$が存在し,
       任意の$0 < \lambda \leq \bar{\lambda}$に対し,
       minimal solution が存在する.
 \item $-\kappa_1 < \kappa \leq 0$のとき,全ての$N \geq 3$について,
       任意の$0 < \lambda < \bar{\lambda}$に対し,
       second solution が存在する.
 \item $\kappa > 0$のとき,$N = 3, 4, 5$については,
       任意の$0 < \lambda < \bar{\lambda}$に対し,
       second solution が存在する.
       一方,$N \geq 6$については,解が一意的である場合がある.
\end{enumerate}
特に3.~は,領域の次元$N$により
解の存在・非存在が異なることを主張している.
本論文は,内藤 -- 佐藤~\cite{MR2886160}の議論を踏襲し,
\ref{eq:prob_main}を考察する.

\ref{eq:prob_main}の minimal solution について以下の定理が従う.
\begin{thm}[本論文 定理~1.1, ~1.2] \label{thm:minimal_solution}
以下の(i) -- (iii)をみたす
 $0 < \bar{\lambda} < \infty$が存在する.
 \begin{enumerate}[(i)]
  \item $0 < \lambda \leq \bar{\lambda}$において,
        \ref{eq:prob_main}は minimal solution を持つ.
  \item $\lambda > \bar{\lambda}$において,\ref{eq:prob_main}は弱解
        を持たない.
  \item $b > 0 ~\tin \Omega$ならば,$\lambda = \bar{\lambda}$に
        おける
        \ref{eq:prob_main}の解は一意的である.
 \end{enumerate}
\end{thm}

定理~\ref{thm:minimal_solution}の証明では,
陰関数定理と minimal solution における線形化固有値問題
\[
-\Delta \phi + a \phi = \mu p b (\underline{u}_\lambda)^{p-1} \phi
  ~\tin \Omega, \ \ \phi \in H_0^1(\Omega)
\] 
が鍵となっている.

\ref{eq:prob_main} の second solution の結果を述べる.
比較のため 内藤 -- 佐藤~\cite{MR2886160}の結果を\ref{eq:prob_main}に
即して述べる.

\begin{thm}[\cite{MR2886160} Theorem 1.3]
 \label{thm:second_solution_naito_sato}
 $0 < \lambda < \bar{\lambda}$とする.
 $\Omega$上$a = \kappa$,$b = 1$は定数とする.
 以下の(i), (ii)の
 いずれかの成立を仮定する.
 \begin{enumerate}[(i)]
  \item $-\kappa_1 < \kappa \leq 0$かつ$N \geq 3$.
  \item $\kappa > 0$かつ$N = 3, 4, 5$.
 \end{enumerate}
 このとき,\ref{eq:prob_main}は,minimal solution
 $\underline{u}_\lambda$
 以外の弱解$\bar{u}_\lambda \in H_0^1(\Omega)$を持ち,
 $\bar{u}_\lambda >
 \underline{u}_\lambda ~\tin \Omega$が成立する.
\end{thm}

本論文では,以下の定理を証明する.

\begin{thm}[本論文 定理~1.4] \label{thm:second_solution}
 $0 < \lambda < \bar{\lambda}$とする.$b$は$\Omega$上の
 ある点$x_0$で最大値$M = \left\| b \right\|_{L^\infty(\Omega)} > 0$を
 達成するものと仮定する.$r_0 > 0$が存在し,
 $\{ \lvert x - x_0 \rvert < 2r_0 \} \subset \Omega$,かつ,
 $\{ \lvert x - x_0 \rvert < r_0 \}$上,$b$は連続であり$a$は
 \begin{equation}
  a(x) = m_1 + m_2 \lvert x-x_0 \rvert^{q} 
  + o(\lvert x-x_0 \rvert^{q}) \ \ \tin \{ \lvert x - x_0 \rvert < r_0
  \}  \label{eq:a_q}
 \end{equation}
 であると仮定する.ここで$q > 0$,
 $m_1 > \kappa$,$m_2 \neq 0$は
 定数である.さらに,以下の(i) -- (iv)の
 いずれかの成立を仮定する.
 \begin{enumerate}[(i)]
  \item $m_1 < 0$,かつ,$N \geq 3$.
  \item $m_1 > 0$,かつ,$N = 3, 4, 5$.
  \item $m_1 = 0$,かつ,$m_2 < 0$,かつ,$N \geq 3$.
  \item $m_1 = 0$,かつ,$m_2 > 0$,かつ,$3 \leq N < 6 + 2q$.
 \end{enumerate}
 このとき,\ref{eq:prob_main}は,minimal solution
 $\underline{u}_\lambda$
 以外の弱解$\bar{u}_\lambda \in H_0^1(\Omega)$を持ち,
 $\bar{u}_\lambda >
 \underline{u}_\lambda ~\tin \Omega$が成立する.
\end{thm}

内藤 -- 佐藤の
定理~\ref{thm:second_solution_naito_sato}では,
$a = \kappa$および$b = 1$であるとき,
$\kappa$が非正から正へと変化すると,
\ref{eq:prob_main}の second solution が存在する
次元$N \geq 3$が,
$N < \infty$から$N < 6$へと
段差的に変化することを示している.
本論文では,$a$は定数ではなく関数であるため,
定理~\ref{thm:second_solution}~(iv) のケースを
検討することができる.$N < 6 + 2q$において
\ref{eq:prob_main}の second solution が存在するという結果は,
「$N < \infty$」と「$N < 6$」の「中間部分」の結果に
相当すると考えられる.
$a$の零点における位数が大きいほど,
second solution が存在する次元も大きくなる.
証明では,
$(\mathrm{PS})$条件を課さない峠の定理と,
タレンティー関数とソボレフ臨界指数の関係が
重要な役割を果たす.

\ref{eq:prob_main}の second solutionが
$(0, \bar{\lambda})$上一様には存在しない場合もある.

\begin{thm}[本論文 定理~1.5] \label{thm:second_solution_nonex}
 \begin{enumerate}[1.]  \sage
  \item $N \geq 3$とする.
        $b > 0 ~\tin \Omega$であると仮定する.
        このとき,$0 < \lambda^* < \bar{\lambda}$が存在し,
        任意の$\lambda^* \leq
        \lambda < \bar{\lambda}$に対し,\ref{eq:prob_main}は
        minimal solution $\underline{u}_\lambda$以外の弱解
        $\bar{u}_\lambda$を持ち,$\bar{u}_\lambda >
        \underline{u}_\lambda ~\tin \Omega$が成立する.
  \item $N \geq 6$とする.$R > 0$とし,
        $\Omega = \{ x \in \R^N \mid \lvert x \rvert < R\}$
        と仮定する.
        $a = a(\lvert x \rvert)$,
        $b = b(\lvert x \rvert)$,
        $f = f(\lvert x \rvert)$は$\Omega$上球対称とする.
        また,$0 < \alpha < 1$とし,$a , b \in C^1([0, R])$,
        $f \in C^\alpha([0, R])$であり,$a$は$[0, R]$上
        単調増加,$b, f$は$[0, R]$上
        単調減少と仮定する.また,$a(0), b(0) > 0$とする.
        このとき,$0 < \lambda_*$が存在し,
        任意の$0 < \lambda < \lambda_*$に対し,
        \ref{eq:prob_main}は
        minimal solution $\underline{u}_\lambda$以外の弱解を持たない.
 \end{enumerate}
\end{thm}

定理~\ref{thm:second_solution_nonex}.2の証明では,
まず,球対称解のみ考慮すれば十分であることを示し,
動径$r$についての常微分方程式に議論を帰着させる.
次に,ポホザエフの議論から不等式を出し,十分小さい$\lambda > 0$では
不等式がみたされないことを示す.

\bibliographystyle{jalpha}
\bibliography{ref}

\end{document}